\documentclass{report}
\usepackage[a4paper, total={6in, 8in}]{geometry}
\usepackage{enumitem}
\setlistdepth{9}

\setlist[itemize,1]{label=$\bullet$}
\setlist[itemize,2]{label=$\bullet$}
\setlist[itemize,3]{label=$\bullet$}
\setlist[itemize,4]{label=$\bullet$}
\setlist[itemize,5]{label=$\bullet$}
\setlist[itemize,6]{label=$\bullet$}
\setlist[itemize,7]{label=$\bullet$}
\setlist[itemize,8]{label=$\bullet$}
\setlist[itemize,9]{label=$\bullet$}

\renewlist{itemize}{itemize}{9}
\usepackage{kotex}
\title{미시경제학 정리}
\date{\today}
\begin{document}
  \maketitle
  \tableofcontents
  \chapter{미시경제학의 주요 주제}
  \begin{itemize}
    \item 엘빈 굴드너
    \begin{itemize}
        \item 경제학에서도 토대가 되는 인간과 사회에 관한 근본적인 가정이 존재할 것이다.
        \item 그 가정의 선택에서 중요한 것은 경제학이 과학이어야 한다는 생각이다.
        \begin{itemize}
            \item 신고전파 경제학의 전통적인 주제를 다루는 이론이 주류경제학의 미시경제학
            \item 미시경제학의 학문적 특성을 이해하는 데 인간과 사회에 관한 가설 및 과학의 본질에 관한 인문학적 논의가 필요
        \end{itemize}
    \end{itemize}
\end{itemize}
\begin{itemize}
    \item 경제학에서도 인간과 사회는 왜 필요한가?
    \begin{itemize}
        \item 시스템을 이룬 분업체계만이 분업의 이득을 실현할 수 있다. (부분들이 법칙적 인간관계에 의해 통합되어 유기적 총체를 구성하고 있는 것을 시스템이라 부른다.)
        \item 인간은 의식적이고 의지적인 존재이다. 그래서 앞서 언급한 분업체계가 시스템의 특성을 갖지 못할 수도 있다.
        \begin{itemize}
            \item 그리하여 인간 역시 멍애로부터 그렇게 자유로울 수 없다는 것을 보여주는 가설이 필요하다.
        \end{itemize}
        \item 가설 선택의 세가지 가능성
        \begin{enumerate}
            \item 인간 본성이 존재하여 사회는 인간(구성원)이 만듦: 인간본성으로 경제체계의 시스템적 안정성을 설명
            \begin{itemize}
                \item 주류 경제학이 선택한 인간과 사회에 관한 설명
            \end{itemize}
            \item 인간 본성의 부재, 사회가 인간을 규정: 개별 인간의 특성이 아닌 것으로 경제체계의 시스템적 안정성을 설명
            \item 인간 본성의 존재, 사회가 인간을 규정: 사회가 비인간적일 수 있고,경제체계의 시스템적 안정성을 보장할 수 없음.
        \end{enumerate}
    \end{itemize}
    \item 과학이 경제학을 무슨 연유에서 어떻게 규정해 왔는가?
    \begin{itemize}
        \item 과학에 대한 경제학의 동정
        \begin{itemize}
            \item 과학이 경제학 발전에 끼친 주요 영향
            \begin{itemize}
                \item 연구방법: 개인이 인간 본성에 따라 연출하는 경제행위를 분석함으로써 경제현상의 과학적 분석이 가능하다는 생각을 심어 줌.
                \item 연구주제: 시장경제가 경제활동에 조화로운 질서를 만들어 낼 수 있다는 것을 논증할 수 있다는 확신을 심어줌
            \end{itemize}
            \item 뉴턴적 세계관과 경제학
            \begin{itemize}
                \item 1687년 발간된 뉴턴의 저서 자연 철학적 수학의 원리에서는 플라톤 - 피타고라스의 세계관과 타당성을 확인시켜줌으로써 자연 세계의 조화로운 질서가 존재한다는 것을 증명하여 경제학의 주제에 결정적인 영향을 끼침.
            \end{itemize}
            \item 신고전파 경제학
            \begin{itemize}
                \item 19세기에는 과학이 주로 분석도구와 분석방법 측면에서 경제학에 영향을 끼치고, 물리학의 분석도구와 분석방법을 공유하는 신고전파 경제학이 탄생하였음
                \item 신고전파 경제학은 연립방정식 형태의 경제모형을 가지고 시장경제체계를 분석할 수 있다는 기능적인 가능성을 제시하였고, 경제분석의 수학의 최적화 기법을 도입할 수 잇다는 것을 제시하였다는 점이 큼
                \item 주류경제학은 20세기에 들어와 이론의 수리화와 계량화를 통하여 경제이론의 과학화를 완성한다는 기획에 매진하였음.
                \item 엘버타대학교에서는 경제학을 흔히 사회과학의 여왕으로 회자한다. 경제학은 권위있는 탐구 주제이다. 경제학은 노벨상이 수상되는 유일한 사회과학이기도 하다. 라고 이야기하고 있다.
            \end{itemize}
        \end{itemize}
        \item 과학의 본질적 특징
        \begin{itemize}
            \item 연구주제를 객관적으로(있는 그대로)인식할 수 있는 유효한 방법을 갖고 있고, 현실을 보편적인 법칙을 사용하여 설명할 수 있어야 한다.
            \item 왜 사물의 객관적인 인식이 어려운가?
            \begin{itemize}
                \item 인식 주체는 사진기처럼 대상의 상태를 수동적으로 기록하는 기계가 아니기 때문이다.
                \item 인식 내용이 인식 주체의 객관적 요인에 영향을 받지 않는 것이 객관적 인식의 필요조건
                \item Charles Van Doren: 과학은 관념이나 감정이 아니라 오로지 사물만을 다룬다.
            \end{itemize}
            \item 관찰과 귀납의 중요성
            \begin{itemize}
                \item 과학적 탐구 방법의 요체: 감각을 전혀 오류가 없는 모든 지식의 원천이라고 보고 귀납, 분석, 관찰, 실험을 합리적 탐구 방법의 주요 조건으로 파악했다.
            \end{itemize}
            \item 논리실증주의적 객관적 인식 방법
            \begin{itemize}
                \item 내용을 경험적 반증이 가능한 진술로 표현하고, 반증되지 않은 진술만을 신뢰함. (과학은 바로 이런 방법을 사용하고 있다.)
                \item 경험적 반증가능성: 경험적 증거를 들어 어떤 진술이 거짓임을 보이는 것이 경험적 반증이다.
                \begin{itemize}
                    \item ``오늘은 열대아가 심하군''이라는 진술은 경험적 반증이 가능함.
                    \item 반증가능성에서 중요한 것은 관찰이다.
                \end{itemize}
                \item 경제현상에서 핵심 매체
                \begin{itemize}
                    \item 기하학적 형상을 지친 물체에 구현되는 현상만을 관찰할 수 있으므로 인간과 재화가 관찰가능한 경제현상의 핵심 매제체이다.
                    \begin{itemize}
                        \item 마르크스 자본론의 방법에서는 경제현상의 핵심 매체인 인간과 상품 중에서 상품을 논의의 출발점으로 선택했다.
                        \item 주류경제학에서의 분석의 출발점은 인간이다. 따라서 모든 논의를 경제주체의 행동을 분석하는 데에서 시작한다.
                    \end{itemize}
                    \item 과학의 목적
                    \begin{itemize}
                        \item 과학의 목적은 직점 경험하는 대신에 사고 속에서 사실을 예견하고 재현함으로써 경험의 수고를 더는 데 있다 - Ernst Mach
                    \end{itemize}
                    \item 과학에서의 설명
                    \begin{itemize}
                        \item 어떤 현상을 설명한다는 것은 그것이 보편 법칙의 한 사례임을 보여주는 데 있다.
                    \end{itemize}
                \end{itemize}
            \end{itemize}
            \item 과학의 세계관이 어떤 것인가에 대한 설명
            \begin{itemize}
                \item 경제학이 과학임을 주장하려면 경제현상의 세계에서도 뉴턴적 세계관이 타당하다고 주장할 수 있어야 한다.
                \begin{itemize}
                    \item 뉴턴적 세계관의 형이상학적 존재들: 유물론, 결정론, 기계론
                    \begin{itemize}
                        \item 유물론: 객관적 실체는 연장을 갖는 물질의 형태로 존재함으로 모든 현상은 물질에 관한 법칙들에 종속되어 있다.
                        \item 결정론: 특정한 과거에 상태로부터 나타날 수 있는 현재의 상태는 단 하나뿐이고, 미래의 특정한 상태를 만들어 낼 수 있는 과거의 상태도 단 하나뿐이다. 이런 관점을 특히 과학적 결정론이라고 한다.
                        \item 기계론: 물질적 세계는 수식으로 정확히 표현될 수 있고, 그 어떤 예외도 허용하지 않는 엄정한 법칙들에 의해 규정된다.
                    \end{itemize}
                \end{itemize}
                \item 주류경제학의 인간관
                \begin{itemize}
                    \item 인간이 자극에 대해 기계적으로 반응하는 것처럼 보이는 다양한 현상들이 존재한다.
                    \begin{itemize}
                        \item 주류경제학자들은 인간의 경제 행위에 법칙성이 존재한다고 생각한다.
                        \item 자극에 대해 예측가능한 반응을 보인다는 것으로 생각이 가능하다.
                    \end{itemize}
                    \item 합리적 행동가설
                    \begin{itemize}
                        \item 인간은 논리정합적인 방식으로 행동이 가능하다면 이기적인 방법도 불사하고 자신의 개인적 이득을 극대화하려고 한다.
                        \begin{itemize}
                            \item 이기적 행동: 자기의 개인적인 목적을 무자비하게 추구하는 것.
                        \end{itemize}
                        \item 행위의 목적과 수단을 명확히 인식하고, 목적 달성하는 절차를 찾아낼 수 있고, 찾아낸 절차를 실행에 옮길 수 있다. (인식, 탐색, 실행의 방법)
                        \item 인긴이 상대적 희소성에 대응하는 근본적인 형식은 선택이다.
                        \begin{itemize}
                            \item 한정된 자원을 가지고 무엇을 얼마만큼 어떻게 만들어 누구에게 공급해야하는지를 결정하는 문제를 존리정합적인 방법으로 해결한다는 의미를 가지고, 앞서 언급한 시스템적 분업체계를 국가적인 체계에서 본다면 자원배분기구라고 할 수 있다.
                        \end{itemize}
                    \end{itemize}
                \end{itemize}
            \end{itemize}
        \end{itemize}
    \end{itemize}
\end{itemize}
\chapter{수요공급모형 I}
\begin{itemize}
    \item 주류경제학은 다음과 같은 일이 주류가 되어야 한다고 생각한다
    \begin{itemize}
        \item 합리적 선택의 보편적인 준칙에 관한 탐구
        \item 시장경제의 구조와 기능을 설명
    \end{itemize}
    \item 교제는 수요공급모형을 시장경제를 설명하는 내용으로 구성되어 있다.
    \begin{itemize}
        \item 수요공급모형의 설명 (제 2장)
        \item 수요공급모형 구성 요소의 분석 (제 3-7장)
        \item 완전히 이해된 수요공급모형 (제 8장)
    \end{itemize}
    \item 완성도가 높은 학문의 주요 특징은 다음과 같다
    \begin{itemize}
        \item 모든 이론의 토대가 되는 직관적인 가설이 존재한다는 것.
        \item 공식화된 소수의 분석 기법을 반복적으로 사용하여 이론을 전개한다는 것
        \begin{itemize}
            \item 뉴턴의 고전역학이론
        \end{itemize}
    \end{itemize}
    \item 주류경제학의 주요 분석기법
    \begin{itemize}
        \item 모형의 의한 분석기법
        \item 균형분석기법
        \item 비교정태 분석기법
        \item 최적화 기법
    \end{itemize}
    \item 모형의 의한 분석기법
    \begin{itemize}
        \item 모형은 도구인가 실제의 기술인가?
        \begin{itemize}
            \item 도구론: 모형은 단지 예측의 도구에 지나지 않는다.
            \item 실제론: 모형은 실제 세계의 본질을 보여주는 지적 구조물이다.
            \begin{itemize}
                \item 모형을 사용하는 목적이 무엇인가?
            \end{itemize}
            \item 과학 이론에서의 설명과 예측: 과학은 보편법칙을 이용해 설명하고 예측함으로써 사유 경제성을 추구한다.
            \begin{itemize}
                \item 설명: 알고있는 현상이 보편 법칙의 구체적인 사례임을 증명
                \item 예측: 보편법칙으로부터 알지 못했던 현상을 발견
            \end{itemize}
            \item 가설-연역체계로서의 모형
            \begin{itemize}
                \item 과학에서는 가설-연역체계가 모형이다
                \item 유클리드는 기원전 300년경에 기하학원론에서 가설-연역체계를 이용한 이론 전개의 시범을 보였음
                \item 아리스토텔레스는 논증과 명제를 설명하였는데, 참다운 지식은 논증적 지식이어야 하므로 학문적 지식의 근간은 논증에 있다. 결론은 그것의 보편성과 진리성을 직관적으로 이해할 수 있는 명제에 근거를 두고 있다는 사실을 보이는 것이 논증의 핵심이다. 라고 말하였다.
            \end{itemize}
            \item 수리모형: 경제현상은 측정 가능한 현상이다.
            \begin{itemize}
                \item 경제모형은 대부분 연립방정식체계의 형태를 취하는 수리모형이다.
            \end{itemize}
        \end{itemize}
        \item 본질과 현상
        \begin{itemize}
            \item 사물의 본질과 현상 형태가 직접적으로 일치한다면 그 어떤 과학도 필요치 않다.
            \item 수요공급모형이 어떤 것인지 알려면 보통 사람들이 생각하는 시장경제의 모형이 어떤 것인지를 알아야 한다.
        \end{itemize}
    \end{itemize}
    \item 시장경제의 수수깨끼
    \begin{itemize}
        \item 시장경제의 현상 형태
        \begin{itemize}
            \item 자유방임
            \item 개인의 고립분산성
        \end{itemize}
        \item 시장경제에서는 무조건적 자유란 존재할 수 없고, 경제활동 조정 장치가 필요하다.
        \begin{itemize}
            \item 시장경제에를 통제하는 능력은 어디에서 나오는가를 분석하려면 수요공급모형을 사용해야 한다.
        \end{itemize}
    \end{itemize}
    \item 수요공급모형의 이론적 착상
    \begin{itemize}
        \item 시민사회의 개념과 법적 개념주의의 관점
        \begin{itemize}
            \item 시민사회란 무엇인가? 국가도 침범할 수 없는 일련의 권리를 행사하는 개인들이 자발적으로 맺는 사적인 사회적 관계를 통해 대부분의 경제활동들이 이루어지는 사회의 영역을 시민사회라 한다.
        \end{itemize}
        \item 시민사회와 법적 개념주의
        \begin{itemize}
            \item 시민사회가 추구하는 자유의 관념과 그것의 사회적 관계를 뒷받힘하는 법적 구속의 관념을 양립시킬 수 있는 방법은 과연 무엇인가? 는 자발적 계약이라고 이야기할 수 있다.
            \begin{itemize}
                \item 주류경제학은 `시민사회'의 사적인 영역을 시장, 계약을 거래 또는 교환, 자발적 계약의 조건을 가격이라고 바꿔 부른다.
                \item 시장경제의 조화로운 경제상태란 경제활동에 관한 자발적 계약에서 모든 사람이 마음에 맞는 계약조건에 도달한 상태를 의미하고, 이를 E 지점이라고 부른다.
            \end{itemize}
            \item 진보적인 사회의 역사는 지금까지 신분에 기초한 사회관계로부터 계약관계로 움직였다. H. Main.
        \end{itemize}
        \item 수요공급모형 착상의 주요 합의
        \begin{itemize}
            \item 시장경제의 수수깨끼를 푸는 열쇠를 가격의 조정기능에서 찾아봐야 한다.
            \begin{itemize}
                \item 사회적 분업을 통일하고 보완하고, 서로 무관한 사람들의 매개인 것
                \item 조정기능을 매개변수적 기능이라고 부른다.
            \end{itemize}
        \end{itemize}
    \end{itemize}
    \item 경제활동의 순환적 흐름
    \begin{itemize}
        \item 시장경제에서도 수많은 사람들의 경제활동이 서로 밀접히 연관되어 있다.
        \item 다양한 경제활동들이 시장에서 가격에 의해 매개되고 조정된다.
    \end{itemize}
    \item 가격의 매개변수적 특징
    \begin{itemize}
        \item 수학에서는 특정 유형의 함수관계를 정의하는 상수를 매개변수이라 한다.
        \item 상수는 그 값이 변할 수 있다는 특정을 가지고 있다.
        \begin{itemize}
            \item 가격이란 가격이 수학적인 파라미터럼 변상수인 것을 말한다.
            \item 개인의 입장에서는 상수이고, 시장에서는 변수인 것이다.
        \end{itemize}
        \item 가격의 매개변수적 특성은 사람들이 가격의 통제를 받으면서도 경제행위의 자유를 누린다는 느낌을 가질 수 있는 필요조건인 것을 말한다.
        \item 특정 개인이 가격에 결정적 영향을 미칠 수 있다면 경제적 자유를 침해당하는 느낌을 받게 된다.
        \begin{itemize}
            \item 가격은 모든 것에 비개인적으로 결정되기 위해서는 시장이 완전경쟁적이어야 한다. 이는 수많은 공급자와 수요자가 있음을 말할 수 있다.
        \end{itemize}
        \item 시장이 완전경쟁적이어서 가격이 매개변수적 특성을 가질 때, 사람들은 가격의 통제를 받으면서도 경제적 자유를 누린다고 느낀다.
    \end{itemize}
    \item 가격의 배분기능
    \begin{itemize}
        \item 자원의 특정한 용처에 대한 사람들의 욕구가 얼마나 강한지 알려주는 신호를 전달함으로써 자원을 유용한 용처로 배분하는 가격의 기능을 가짐.
    \end{itemize}
    \item 가격의 배급기능
    \begin{itemize}
        \item 재화의 가치를 보다 높게 평가하는 사람들의 순서에 따라 재화를 배급하여 희소한 재화를 누가 사용할지 결정하게 된다.
    \end{itemize}
    \item 시장에서의 균형이란 무엇인가?
    \begin{itemize}
        \item 균형: 개인들 모두가 경제활동에서 자신이 의도하는 바를 모두 실현할 수 있는 상태를 말한다. 모든 소비자는 원하는 수량만큼 소유할 수 있고, 모든 생산자는 생산하고 싶은 양을 모두 생산할 수 있는 상태를 의미할 수 있다.
    \end{itemize}
    \item 수요공급모형의 핵심요소
    \begin{itemize}
        \item 수요자와 공급자의 마음을 보여주는 것이 수요곡선과 공급곡선이다.
        \item 개별수요곡선: 시장각겨이 개발소비자의 수요량을 결정하는 함수적 관계를 표현하는 방법.
        \item 시장수요곡선: 시장가격이 변화함에 따라 시장수요량이 어떻게 바뀌는지를 보여주는 방법
        \item 수요곡선
        \begin{itemize}
            \item 수학에서는 함수관계를 그래프로 나타낼 때 원인변수인 종속변수를 가로축에, 결과변수인 종속변수를 세로축에 표현하는 것이 관례이나 수요곡선인 경우에는 독립변수를 가로축에 종속변수를 세로축에 표시함.
            \item 종속변수에 시장 전체의 수요량이 표시되면 시장수요곡선이고, 독립적인 소비자의 개별수요량을 나타내면 개별수요곡선인 것이다.
        \end{itemize}
        \item 수요표와 수요함수
        \begin{itemize}
            \item 수요곡선은 몇개의 점을 뽑아서 표로 제시한 것을 의미
            \begin{itemize}
                \item 그러나 모든 것이 아닌 몇개의 점을 뽑아서 표시한 것이기 때문에 알 수 없음.
            \end{itemize}
            \item 수요함수는 수요량을 함수화하여 포시한 것을 의미
        \end{itemize}
        \item 수요법칙
        \begin{itemize}
            \item 수요법칙: 가격이 쌀 수록 수요량은 늘어날 것이고, 가격이 비쌀 수록 수요량은 줄어들 것.
            \item 시장수요곡선에서 수요곡선 기울기가 우하향하고, 시장수요함수에서 기울기 상수가 음(-)인 것은 모두 앞서 언급한 수요법칙에 의해서 발생하기 때문이다.
        \end{itemize}
        \item 공급곡선
        \begin{itemize}
            \item 공급자의 마음에서 가격이 얼마에 도달할 경우 얼마나 공급할지에 대한 것.
            \item 공급법칙: 시장공급량 또한 가격이 오르면 증가할 것이고, 가격이 내리면 감소할 것.
        \end{itemize}
    \end{itemize}
    \item 수요공급모형과 균형분석
    \begin{itemize}
        \item 균형의 존재 문제: 그래프를 참고하면 시장공급곡선과 시장수요곡선이 만나는 점을 E 점이라고 하고 이 점을 중심으로 수요와 공급량이 결정된다.
        \item 균형의 안정성 문제
        \begin{itemize}
            \item 가격이 매개변수적 특성을 갖는 수요공급모형에서도 시장가격을 변경할 의도가 없이 이루어지는 수요자와 공급자의 호가변경을 통해 불균형이 조정된다.
            \item 초과수요
            \begin{itemize}
                \item 초과수요가 존재한다는 것은 주어진 가격에서 실제로 구입할 수 없다는 소비자들이 존재한다는 뜻이다.
                \begin{itemize}
                    \item 비싼 가격을 지불해서 구매할 사람들은 가격을 올릴 마음이 없겠지만 그런 사람들이 모여들어 호가를 올릴 것이고 그것들이 바로 초과수요가 생기는 것이다.
                \end{itemize}
                \item 공급자는 바로 이런 초과수요를 알아차리고 공급량을 늘림으로써 초과수요는 다시 균형상태로 돌아올 것이다.
            \end{itemize}
            \item 초과공급: 공급량이 수요량보다 많아서 도저히 이 가격에 물건을 팔 수 없을 경우에 호가를 낮춤으로써 불균형상태에 있던 가격이 다시 균형상태의 가격으로 다시 돌아로 것이다.
        \end{itemize}
        \item 결론: 초과수요나 초과공급이 일어난다 하더라도 이것들은 점차 균형상태로 돌아올 것이다.
    \end{itemize}
\end{itemize}
\chapter{수요공급모형 II}
\begin{itemize}
    \item 비교정태분석
    \begin{itemize}
        \item 비교정태분석 방법론적 중요성
        \begin{itemize}
            \item 비교정태분석이 모형에 의한 분석의 백미이다.
            \item 모형에 의한 분석은 현상을 논리적으로 재연하는것을 의미한다.
            \begin{itemize}
                \item 아리스토텔레스는 올바은 연구방법의 핵심을 다음과 같이 설명하였는데, 설명하려는 현상으로부터 설명 원리를 귀납하고, 그 다음에는 현상에 대한 명제를 그 원리로부터 다시 연구해야 한다고 하였음.
                \item 뉴턴의 방법
                \begin{itemize}
                    \item 관찰 자료로부터 경험법칙을 도출
                    \item 모형으로 경험법칙을 논리적으로 재연
                    \item 모형의 타당성을 추가로 검정
                    \item 모형으로 미지 현상을 예측
                    \begin{itemize}
                        \item 예측이 과학적 이론의 궁극적 목적
                        \item 고전역학을 이용한 혜왕성의 발견은 과학의 궁극적인 유용성이 예측에 있다는 사실을 극적으로 보여주고 있음.
                    \end{itemize}
                \end{itemize}
            \item 경제모형의 분석에서 비교정태분석이란 무엇인가?
            \begin{itemize}
                \item 경제모형에서 에측 형태를 취한 현상의 이론적 재연을 이끌어내는 분석을 의미한다.
            \end{itemize}
        \end{itemize}
    \end{itemize}
    \item 방정식에 의한 수요공급모형의 표현
    \begin{itemize}
        \item $Supply=c-dp$
        \item $Demand=a+bp$
        \item $Equal = S() == D()$
        \item 수요, 공급, p : 변수
        \item a, b, c, d : 상수
        \begin{itemize}
            \item 수요와 공급에 영향을 미치는 요인이 상품가격뿐이면 비교정태분석은 불가능하다.
        \end{itemize}
    \end{itemize}
    \item 일반적 수요함수
    \begin{itemize}
        \item $Q_d$ = f(가격, 기호, 소득, 대체제의 가격, 보완재의 가격, 소비자들의 기대)
        \item $Q_d = f(p_x, t_x, I, p_y, p_z, v_d)$
    \end{itemize}
    \item 수요의 여타 결정 요인
    \begin{itemize}
        \item 소비자의 기호
        \item 소비자의 소득
        \begin{itemize}
            \item 소득은 수요량에 영향을 미친다
            \item 정상재: 소득이 증가(감소)하면 수요량도 증가(감소)하는 재화
            \item 열등재: 소득이 증가(감소)하면 수요량이 오히려 감소(증가)하는 재화
            \begin{itemize}
                \item 보리나 감자와 같은 경우를 열등재라고 한다.
            \end{itemize}
            \item 대체재: 용도상 X재를 대신할 수 있는 재화
            \item 보완재: X재가 제 기능을 발휘하기 위해 반드시 필요한 재화
        \end{itemize}
        \item 관련 재화의 가격
        \item X재의 미래 가격에 대한 소비자들의 기대
    \end{itemize}
    \item 공급의 여타 주요 결정 요인
    \begin{itemize}
        \item 생산기술
        \item 생산요소의 가격
        \item 생산물 미래 가격에 대한 공급자 기대
    \end{itemize}
    \item 일반적 공급함수
    \begin{itemize}
        \item $Q_s = g(p_x, \alpha, r, w, v_s)$
        \item 공급 = g(가격, 생산기술, 생산요소, 생산요소 K의 가격, 생산요소 L의 가격, 미래 가격에 대한 생산자들의 기대)
    \end{itemize}
    \item 내생변수와 외생변수
    \begin{itemize}
        \item 내생변수: 가격, 수요량, 공급량
        \item 외생변수: 내생변수를 제외한 모두
        \item 수요공급모형에서 외생변수의 변화는 수요곡선이나 공급곡선 자체를 변화시킨다.
    \end{itemize}
    \end{itemize}
    \item 수요공급모형의 수치 예
    \begin{itemize}
        \item 수요함수: $Q_d = 20-2p_x+0.5p_y+0.4I$
        \item 공급함수: $Q_d = 10+0.5p_x-0.2w$
        \item 균형조건식: $Q_d = Q_s$
    \end{itemize}
    \item 비교정태분석이란?
    \begin{enumerate}
        \item 경제모형에서 특정 외생변수의 값이 변함에 따라 균형이 어떻게 달라지는지 알아보는 분석
        \item 경제모형에서 예측 형태를 취한 현상의 이론적 재현을 이끌어내는 분석
    \end{enumerate}
    \item 수요량 변화의 두 형태
    \begin{itemize}
        \item 외생변수가 변화하는 경우: 수요곡선 자체를 왼쪽 또는 오른쪽으로 변화시킴. (수요의 변화)
        \item 내생변수가 변화하는 경우: 내생변수의 변화로 인한 수요곡선 상의 이동. (수요량의 변화)
    \end{itemize}
    \item 다양한 비교정태분석 예시
    \begin{enumerate}
        \item 보완재 가격이 상승: 보완재 가격의 상승은 균형가격을 하락시키고 균형거래량 또한 감소시킨다.
        \item 유류세 인하 효과: 유류세 인하는 균형가격을 하락시키고 균형거래량을 증가시킨다.
    \end{enumerate}
    \item 가격의 조정 능력은?
    \begin{itemize}
        \item 경제문제 해결에 가격 조정 능력의 측정이 필요할 수 있다.
        \item 탄력성: 함수에서 독립변수에 대해 종속변수가 어느정도 민감하게 반응하는지 측정하는 도구가 \textbf{탄력성}이다.
        \begin{itemize}
            \item 이를 수요(공급)함수에서 독립변수들 중 상품 자체 가격에 대해 측정한 탄력성이 \textbf{수요(공급)의 가격탄력성}이라고 한다.
        \end{itemize}
        \item 수요가격탄력성의 개념적 정의
        \begin{itemize}
            \item $$n_x = -\frac{\frac{\Delta Q_d}{Q_d}}{\frac{\Delta p_x}{p_x}} = -\frac{\%\Delta Q_d}{\%\Delta p_x}$$
        \end{itemize}
        \item 탄력성 공식의 주의점
        \begin{enumerate}
            \item 가격 증분에 대한 수요량 증분의 비울이 아니라 가격 증분의 백분율에 대한 수요량 증분의 배분율의 비율로 탄력성을 계산
            \item 앞서 본 `수요가격탄력성의 개념적 정의' 그래프를 보면 왼쪽에 있는 식 앞에 음($-$)의 부호를 붙여주고 있지만, 수요법칙 때문에 오른쪽 수식에 붙인 음($-$)의 기호만 보고 값이 마이너스임을 판단해선 안됨.
            \item 호탄력성과 점탄력성
            \begin{itemize}
                \item 호탄력성: 변수의 증분이 비교적 클 때 사용하는 탄력성 계산공식
                \item 점탄력성: 변수의 증분이 0에 가까운 미소값일 때 사용하는 탄력성 계산 공식
            \end{itemize}
        \end{enumerate}
    \end{itemize}
\end{itemize}
\chapter{소비자선호와 효용함수}
\begin{itemize}
    \item 수요법칙이란 무엇인가?
    \begin{itemize}
        \item 수요법칙을 효용극대화모형을 가지고 이론적으로 재현하는 것이다.
        \item 합리적 행동모형: 소비자 선택문제를 합리적 행동가설로 정식화한 모형.
    \end{itemize}
    \item 제 4장의 주요 주제
    \begin{enumerate}
        \item 수요이론의 방법론적 특징
        \item 효용극대화모형의 설명
        \item 로빈스 선택문제의 전범으로서의 효용극대화모형
        \begin{itemize}
            \item 4강에서는 여기까지만 다룸
        \end{itemize}
        \item 효용함수의 논리적 타당성
    \end{enumerate}
    \item 수요이론의 방법론적 특징
    \begin{enumerate}
        \item 수요법칙을 개인의 경제행위로부터 설명하는 것
        \item 최적화모형으로 수요이론을 전개
    \end{enumerate}
    \item 합리적 행동가설의 방법론적 함의
    \begin{itemize}
        \item 합리적 행동가설롤 경제현상을 설명
        \item 인간본성의 기계론적인 연출에 의거하여 경제현상을 설명
        \item 사회 차원의 경제현상을 개인의 합리적 선택 행위의 결과로 설명
        \begin{itemize}
            \item ``사회 파원의 경제현상을 개인의 합리적 선택 행위의 결과로 설명하는 것을'' \textbf{방법론적 개인주의}라 한다.
        \end{itemize}
    \end{itemize}
    \item 홉스의 연구 방법
    \begin{itemize}
        \item 방법론적 개인주의는 경제학자가 아닌 정치학자였던 홉스가 만든 방법.
        \item 홉스는 인간 본성의 관한 가설을 가지고 \textbf{시민사회의 기원과 본질}을 설명하려고 시도하였음, 방법은 간단하였는데 전체를 부분으로 쪼개 기본적 요소를 분리해내서 찾아낸 부분의 운동 법칙을 가지고 전체를 설명하는 방법을 사용하였음.
        \item 홉스의 접근방법의 원천
        \begin{itemize}
            \item 홉스의 방법론적 개인주의는 케플러와 갈릴레이의 물리 이론 접근방법인 \textbf{환원주의}를 사용하였음.
            \item 환원주의는 물리적 체계를 구성하는 기본적인 단위를 고찰하는 것을 물리 현상의 올바른 연구 방법이라고 보는 것이라 말하고 물질적 입자들로의 물리적 체계의 분해 가능성과 창발적 특성의 부재를 전제하고 있음.
        \end{itemize}
        \item 방법론적 개인주의
        \begin{itemize}
            \item 사회적 현상의 경우에도 물리적 현상의 경우처럼 창발적 특성이란 존재하지 않는다고 가정하였고 이는 사회를 구성하는 \textbf{개별자}를 분석해야 사회현상이 가장 잘 설명됨을 알려주고 있다.
            \item 주류경제학은 \textbf{개인}을 평가, 선택, 행위의 궁극적인 단위로 삼고 분석을 시작한다. 결과를 창출하는 과정이나 제도 구조가 아무리 복잡하더라도 \textbf{개인의 선택}에 분석 초점을 맞춘다.
            \begin{itemize}
                \item 수학의 관점에서 보면, 최적화가 합리적 행동 가설의 핵심이라고 할 수 있고 이를 경제문제를 최적화모형으로 정식화해 최적화 기법으로 분석할 수 있다.
                \begin{itemize}
                    \item 그리고 이렇게 최적화 기법을 사용하여 분석하는 일을 획기적인 사건으로 보는 용어로 한계혁명이라고 부르기도 하고 있다.
                    \item 이에 밀턴 프리드먼(Milton Friedman)은 ``케인즈 거시이론이 방법론적으로 타당하려면 모형의 모든 가정들은, 비록 그것이 거시적 관점이라고 할지라도, 개인들의 극대화 행동 양식을 가지고 설명될 수 있는 것들이어야 한다. 만약 그렇지 않다면 자의적 방법이므로 폐기되어야 한다''라고 악평을 하였고, 다른 경제학자 들 중에서도 악평을 퍼붓기도 하였다.
                \end{itemize}
            \end{itemize}
        \end{itemize}
        \item 최저화모형으로서의 효용극대화모형
        \begin{itemize}
            \item 합리적 행동가설에 따라 행동하는 소비자의 선택 문제를 \textbf{최적화모형}으로 정식화한것.
            \item \textbf{목적과 수단을 수학적으로 나타내는 요소}가 최적화모형의 필수 구성 요소이다.
            \item 
        \end{itemize}
    \end{itemize}
\end{itemize}
\end{document}